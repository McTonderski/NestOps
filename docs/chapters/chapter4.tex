\chapter{Implementacja systemu}

\section{Backend - API do zarządzania systemem}

\subsection{Struktura i kluczowe endpointy API}

\subsection{Obsługa uwierzytelniania i autoryzacji}

\section{Frontend - Interfejs uzytkownika}

\subsection{Projekt UI/UX}

\subsection{Implementacja aplikacji webowej}

\section{Automatyzacja Konfiguracji i wdrozenie}

\subsection{Integracja z narzędziami CI/CD}
\label{sec:integracja_ci_cd}

Współczesne systemy informatyczne wymagają nie tylko solidnej implementacji, ale również efektywnego zarządzania cyklem życia oprogramowania. W tym kontekście, integracja narzędzi CI/CD (Continuous Integration / Continuous Deployment) odgrywa kluczową rolę w automatyzacji procesów budowania, testowania i wdrażania aplikacji.

\subsubsection{Zastosowanie self-hosted runnera}

W celu zapewnienia kompatybilności architektury uruchomieniowej systemu, zdecydowano się na użycie **self-hosted runnera** zamiast domyślnych runnerów GitHub Actions. Domyślne maszyny CI/CD oferowane przez GitHub działają wyłącznie na **architekturze x86**, co ogranicza możliwość testowania i wdrażania aplikacji na innych platformach, takich jak **ARM** (np. Raspberry Pi, serwery oparte na ARM64). 

Zastosowanie własnego runnera pozwala na:
\begin{itemize}
    \item Uruchamianie testów i budowanie obrazów Docker na architekturze zgodnej z docelowym środowiskiem produkcyjnym.
    \item Pełną kontrolę nad zasobami sprzętowymi wykorzystywanymi w procesie CI/CD.
    \item Możliwość integracji z lokalnym registry dla przechowywania obrazów Docker.
\end{itemize}

\subsubsection{Workflow GitHub Actions dla testowania}

\begin{verbatim}
name: Run Tests

on:
  push:
    branches:
      - main
      - dev
  pull_request:

jobs:
  test:
    runs-on: self-hosted

    services:
      mongo:
        image: mongo:latest
        ports:
          - 27017:27017

    steps:
      - name: Checkout Repository
        uses: actions/checkout@v4

      - name: Set Up Python
        uses: actions/setup-python@v4
        with:
          python-version: "3.13"

      - name: Install Dependencies
        run: |
          python -m pip install --upgrade pip
          pip install -r requirements.txt
          pip install pytest pytest-asyncio httpx

      - name: Run Pytest
        run: pytest -v
\end{verbatim}

\subsubsection{Workflow GitHub Actions dla budowania i wersjonowania obrazów Docker}

Po przejściu testów jednostkowych obraz Docker jest budowany i pushowany do lokalnego rejestru uruchomionego na **localhost:5000**.

\begin{verbatim}
name: Build and Push Docker Image

on:
    push:
      branches:
        - main
        - dev
    pull_request:
        branches: [dev, main]

jobs:
  build_and_push:
    runs-on: self-hosted

    steps:
      - name: Checkout Repository
        uses: actions/checkout@v4

      - name: Extract Version
        run: echo "VERSION=$(date +'%Y%m%d')-$(git rev-parse --short HEAD)" >> $GITHUB_ENV

      - name: Build Docker Image
        run: |
          docker build -t localhost:5000/myapp:${{ env.VERSION }} .
          docker tag localhost:5000/myapp:${{ env.VERSION }} localhost:5000/myapp:latest

      - name: Push Docker Image to Local Registry
        run: |
          docker push localhost:5000/myapp:${{ env.VERSION }}
          docker push localhost:5000/myapp:latest
\end{verbatim}

\subsubsection{Korzyści z bezpośredniego pushowania obrazu do registry}

W przeciwieństwie do wcześniejszej konfiguracji, w której obraz był zapisywany lokalnie za pomocą \texttt{docker save}, obecnie jest on od razu przesyłany do prywatnego rejestru. Takie podejście:
\begin{itemize}
    \item Umożliwia natychmiastowe wykorzystanie obrazu w środowisku produkcyjnym bez potrzeby ręcznego jego ładowania.
    \item Pozwala na prostsze zarządzanie wersjami obrazów w registry.
    \item Minimalizuje czas między budowaniem a wdrożeniem.
\end{itemize}

\subsubsection{Publikacja obrazu do lokalnego registry}

Wszystkie wersje aplikacji są automatycznie przechowywane w lokalnym rejestrze **Docker Registry**, a najnowsza wersja jest oznaczana jako \texttt{latest}. Dzięki temu wdrożenie nowej wersji sprowadza się do uruchomienia nowego kontenera:

\begin{verbatim}
docker pull localhost:5000/myapp:latest
docker run -d --name myapp localhost:5000/myapp:latest
\end{verbatim}

Dzięki temu każda wersja aplikacji jest jednoznacznie identyfikowana, a \texttt{latest} wskazuje na najnowszą stabilną wersję.
\chapter{Czym jest HomeLab oraz analiza istniejących rozwiązań}

\section{Definicja HomeLab oraz znaczenie}
Homelab jest prywatnym środowiskiem IT, dzięki któremu entuzjaści nowych technologii, administratorzy systemów oraz programiści mogą w lokalnym - domowym środowisku testować, rozwijać oraz zarządzać własną infrastrukturą IT. Jego głównym zamierzeniem jest stworzenie realistycznego środowiska do eksperymentowania z technologiami cloudowymi, wirtualizacją, konteneryzacją oraz narzędziami DevOps. Własny system Homelab to równiez metoda na rezygnację z komercyjnych subskrypcji, takich jak Google Drive, DropBox czy OneDrive, co pozwala na w pełni kontrolowanie kto ma dostęp do naszych prywatnych danych. Dzięki niemu zwiększa się prywatność poprzez wyeliminowanie potrzeby przechowywania zdjęć w usługach chmurowych takich jak Google Photos.
Homelaby znajdują zastosowanie w wielu obszarach, w tym:
\begin{itemize}
    \item nauka administracji serwerami i sieciami,
    \item testowaniu nowych technologii przed uzyciem jej w środowisku produkcyjnym,
    \item budowaniu prywatnej chmury oraz rozwiązań do przechowywania danych,
    \item analizie bezpieczeństwa i przeprowadzaniu testów penetracyjnych,
    \item tworzenie autoamtyzacji dla infrastruktury IT,
    \item uniezaleznieniu się od komercyjnych dostawców chmury w celu zwiększenia kontroli nad własnymi danymi.
\end{itemize}

\section{Technologie wykorzystywane w homelabach}
Homelab moze składać się z róznych komponentów, od dedykowanych serwerów fizycznych po rozwiązania chmurowe i kontenerwowe. Kluczowe technologie wykorzystywane w homelabach obejmują:

\subsection{Wirtualizacja i konteneryzacja}
\begin{itemize}
    \item Proxmox VE - platforma do zarządzanai maszynami wirtualnymi i kontenerami.
    \item VMware ESXi - profesjonalne narzędzie do wirtualizacji serwerów.
    \item Hyper-V - narzędzie do wirtualizacji dostarczane przez Microsoft wraz z systemem Windows.
    \item Docker i Kubernetes - technologie konteneryzacji, pozwalające na elsatyczne zarządzanie aplikacjami i zasobami.
\end{itemize}

\subsection{Automatyzacja i zarządzanie konfiguracją}

\begin{itemize}
    \item Ansible, Terraform, Puppet, Chef - narzędzia do automatyzacji wdrazania i zarządzania infrastrukturą.
\end{itemize}

\subsection{Monitoring i analiza}
\begin{itemize}
    \item Prometheus i grafana - rozwiązanai do monitorowania wydajności i wizualizacji danych.
    \item Zabbix - platforma do monitorowania infrastruktury IT.
\end{itemize}

\section{Analiza istniejących systemów do zarządzania homelabem}
\subsection{Przegląd dostępnych rozwiązań}
Na rynku istnieje kilka systemów umozliwiajacych zarzadzanie homelabem. Do najpopularniejszych nalezą:
\begin{itemize}
    \item Proxmox VE - rozbudowany, open-source rozwiązanie do zarządzania maszynami wirtualnymi i kontenerami, oferujące intrgeację z Ceph i wysoką dostępność.
    \item Unraid - popularne rozwiązanie NAS z obsługą wirtualizacji i kontenerów, cenione za łatwość obsługi ale ograniczone zastosowanie korporacyjne.
    \item OpenStack - potęzna platforma chmurowa, która moze być uzywana do zarządzania homelabem, ale jej skomplikowana konfiguracja sprawia, ze nie jest przyjazna dla poczatkujacych uzytkowników.
    \item TrueNAS - rozbudowane oprogramowanie do zarządzania przestrzenią dyskową, które umozliwa tworzenie prywatnych chmur danych
    \item Docker + Kubernetes - stosowane w bardziej zaawansowanych wdrozeniach do zarzadzania kontenerami, ale wymagające większej wiedzy technicznej.
\end{itemize}
\subsection{Zalety i ograniczenia konkurencyjnych systemów}

\subsubsection{Proxmox VE}
\begin{minipage}{0.45\textwidth}
    Zalety
    \begin{itemize}
        \item Darmowa wersja open-source.
        \item Wsparcie dla maszyn wirtualnych (KVM) i kontenerów (LXC).
        \item Mozliwość tworzenia klastrów wysokiej dostępności.
    \end{itemize}
\end{minipage}\hfil
\begin{minipage}{0.45\textwidth}
    Wady
    \begin{itemize}
        \item Brak pełnej automatyzacji wdrozeń.
        \item Stosunkowo wysoki próg wejścia dla początkujących uzytkowników.
    \end{itemize}
\end{minipage}

\subsubsection{Unraid}
\begin{minipage}{0.45\textwidth}
    Zalety
    \begin{itemize}
        \item Intuicyjny interfejs uzytkownika.
        \item Łatwa obsługa pamięci masowej i kontrolerów.
    \end{itemize}
\end{minipage}\hfil
\begin{minipage}{0.45\textwidth}
    Wady
    \begin{itemize}
        \item Model licencyjny oparty na opłacie jednorazowej.
        \item Ograniczona integracja z systemami chmurowymi.
    \end{itemize}
\end{minipage}

\subsubsection{OpenStack}
\begin{minipage}{0.45\textwidth}
    Zalety
    \begin{itemize}
        \item Zaawansowane funkcje chmurowe.
        \item Skalowalność i modularność.
    \end{itemize}
\end{minipage}\hfil
\begin{minipage}{0.45\textwidth}
    Wady
    \begin{itemize}
        \item Bardzo wysoka trudność wdrozenia.
        \item Wymaga duzej ilości zasobów sprzętowych.
    \end{itemize}
\end{minipage}

\subsubsection{TrueNAS}
\begin{minipage}{0.45\textwidth}
    Zalety
    \begin{itemize}
        \item Silne wsparcie dla przechowywania danych.
        \item Wbudowana replikacja i ochrona RAID
    \end{itemize}
\end{minipage}\hfil
\begin{minipage}{0.45\textwidth}
    Wady
    \begin{itemize}
        \item Skupione głównie na funkcjach NAS.
        \item Brak natywnego wsparcia dla maszyn wirtualnych.
    \end{itemize}
\end{minipage}


\subsubsection{Docker + Kubernetes}
\begin{minipage}{0.45\textwidth}
    Zalety
    \begin{itemize}
        \item Elastyczność w zarządzaniu aplikacjami kontenerowymi.
        \item Łatwe skalowanie infrastruktury.
    \end{itemize}
\end{minipage}\hfil
\begin{minipage}{0.45\textwidth}
    Wady
    \begin{itemize}
        \item Wymaga duzej wiedzy technicznej.
        \item Brak wsparcia dla maszyn wirtualnych.
    \end{itemize}
\end{minipage}


\subsection{Identyfikacja luki technologicznej}
Analiza powyzszego porównania dostępnych systemów pokazuję, ze zadne z obecnyhc rozwiązań nie zapewnia jednocześnie:
\begin{itemize}
    \item Pełnej integracji zarządzanai maszynami wirtualnymi, kontenerami i przestrzenią dyskową w jednym ekosystemie.
    \item Prostego i intuicyjnego interfejsu dla uzytkownikow niebędących ekspertami w zarządzaniu infrastrukturą IT.
    \item Natychmiastowej automatyzacji wdrazania, bez konieczności skomplikowanej konfiguracji narzędzi DevOps.
    \item Wbudowanej funkcjonalności związanej z bezpieczeństwem i prywatnością, eliminującej konieczność korzystania z komercyjnych rozwiązań chmurowych.
\end{itemize}

Proponowany system HomeLab ma nacelu uzupełnienie tej luki poprzez stworzenie intuicyjnego narzędzia do zarządzania domową infrastrukturą IT, które zapewni łatowść obsługi, pełną automatyzację oraz zwiększoną przywatność uzytkowników.
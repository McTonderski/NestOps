\chapter{Czym jest HomeLab oraz analiza istniejących rozwiązań}

\section{Definicja HomeLab oraz znaczenie}
Homelab jest prywatnym środowiskiem IT, dzięki któremu entuzjaści nowych technologii, administratorzy systemów oraz programiści mogą w lokalnym - domowym środowisku testować, rozwijać oraz zarządzać własną infrastrukturą IT. Jego głównym zamierzeniem jest stworzenie realistycznego środowiska do eksperymentowania z technologiami cloudowymi, wirtualizacją, konteneryzacją oraz narzędziami DevOps. Własny system Homelab to równiez metoda na rezygnację z komercyjnych subskrypcji, takich jak Google Drive, DropBox czy OneDrive, co pozwala na w pełni kontrolowanie kto ma dostęp do naszych prywatnych danych. Dzięki niemu zwiększa się prywatność poprzez wyeliminowanie potrzeby przechowywania zdjęć w usługach chmurowych takich jak Google Photos.
Homelaby znajdują zastosowanie w wielu obszarach, w tym:
\begin{itemize}
    \item nauka administracji serwerami i sieciami,
    \item testowaniu nowych technologii przed uzyciem jej w środowisku produkcyjnym,
    \item budowaniu prywatnej chmury oraz rozwiązań do przechowywania danych,
    \item analizie bezpieczeństwa i przeprowadzaniu testów penetracyjnych,
    \item tworzenie autoamtyzacji dla infrastruktury IT,
    \item uniezaleznieniu się od komercyjnych dostawców chmury w celu zwiększenia kontroli nad własnymi danymi.
\end{itemize}

\section{Technologie wykorzystywane w homelabach}
Homelab moe składać się z róznych komponentów, od dedykowanych serwerów fizycznych po rozwiązania chmurowe i kontenerwowe. Kluczowe technologie wykorzystywane w homelabach obejmują:

\subsection{Wirtualizacja i konteneryzacja}
\begin{itemize}
    \item Proxmox VE - platforma do zarządzanai maszynami wirtualnymi i kontenerami.
    \item VMware ESXi - profesjonalne narzędzie do wirtualizacji serwerów.
    \item Hyper-V - narzędzie do wirtualizacji dostarczane przez Microsoft wraz z systemem Windows.
    \item Docker i Kubernetes - technologie konteneryzacji, pozwalające na elsatyczne zarządzanie aplikacjami i zasobami.
\end{itemize}

\subsection{Automatyzacja i zarządzanie konfiguracją}

\begin{itemize}
    \item Ansible, Terraform, Puppet, Chef - narzędzia do automatyzacji wdrazania i zarządzania infrastrukturą.
\end{itemize}

\subsection{Monitoring i analiza}
\begin{itemize}
    \item Prometheus i grafana - rozwiązanai do monitorowania wydajności i wizualizacji danych.
    \item Zabbix - platforma do monitorowania infrastruktury IT.
\end{itemize}
\chapter{Projekt Systemy Homelab}

\section{Wymagania funkcjonalne i niefunkcjonalne}

\subsection{Wymagania funkcjonalne}

\begin{enumerate}
    \item Zarządzanie infrastrukturą
    \item Panel administracyjny
    \item Baza danych
    \item Bezpieczny dostęp zdalny
    \item Automatyzacja wdrozeń
    \item Obsługa domeny
    \item Monitorowanie zasobów
    \item Wsparcie dla rozszerzeń
    \item Łatwe wdrazanie aplikacji
    \item Bezpieczne uwierzytelnianie i autoamtyzacja
\end{enumerate}

\subsection{Wymagania niefunkcjonalne}

\begin{enumerate}
    \item Niski pobór energii - system wdrazany na Raspberry Pi 5, co zapeni efektywność energetyczną.
    \item Wysoka dostępność - redundancja i odporność na awarię dzięki Docker oraz Integracji z VPN.
    \item Łatwość w utrzymaniu - system powinien umozliwiac łatwe aktualizację i rekonfigurację w razie potrzeby ręcznej interwencji.
    \item Skalowalność - mozliwość rozszerzenia o nowe komponenty i usługi.
    \item Bezpieczeństwo - szyfrowanie komunikacji oraz kontrola dostępu do zasobów.
    \item Modularność - podział systemu na niezalezne komponenty działające w kontenerach Docker.
    \item Integracja z open-source - Wspracie dla narzędzi i trchnologii dostępnych na licencji open-source.
    \item Minimalizacja kosztów - niskie koszty sprzętowe i utrzymanie dzieki Raspberry Pi i rozwiązaniom chmurowym typu DuckDNS.
    \item Wydajność - optymalizacja aplikacji pod Raspberry Pi, aby zapenić płynne działąnie 
    \item Łatwość wdrozenia - uproszczona konfiguracja pozwalająca na szybkie uruchomienie systemu.
\end{enumerate}

\section{Architektura systemu}
System HomeLab składa się z kilku kluczowych komponentów:

\subsection{Backend (FastAPI + MongoDB)}

\subsection{Frontend (AppSmith)}

\subsection{Warstwa sieciowa}

\subsection{Środowisko kontenerowe}

\subsection{Automatyzacja CI/CD}

\subsection{Urządzenie docelowe}

\begin{itemize}
    \item Raspberry Pi 5
\end{itemize}



\section{Technologie i narzędzia uzyte w systemie}

System HomeLab wykorzystuje następujące technologie:
\subsection{Backend}
\begin{itemize}
    \item FastAPI - szybki i nowoczesny framework do tworzenia API w pythonie
    \item MongoDB - baza danych NoSQL przechowująca konfigurację i dane uzytkowników
\end{itemize}
\subsection{Frontend}
\begin{itemize}
    \item AppSmith - niskokodowe narzędzia do budowy interfejsu uzytkownika.
    \item RestAPI - wykorzystywane do komunikacji między frontendem a backendem.
\end{itemize}
\subsection{Warstwa Sieciowa}
\begin{itemize}
    \item Tailscale - VPN do bezpiecznego zapewnienia zdalnego dostępu do systemu, bez konieczności posiadania stałego adresu IP.
    \item DuckDNS - dynamiczny system zarządzania domeną umozliwiający łatwy dostęp do systemi.
\end{itemize}
\subsection{Środowisko uruchomieniowe}
\begin{itemize}
    \item Docker - uzywany do konteneryzacji aplikacji i zarządzania zaleznościami.
    \item Raspberry Pi 5 - host systemu zapewniający energooszczędność i niski koszt.
\end{itemize}
\subsection{Automatyzacjia CI/CD}
\begin{itemize}
    \item GitHub Actions - narzędzie do automatyzacji wdrozeń i testowania kodu.
    \item Pipeline CI/CD - automatyczne testowanie, budowanie i wdrazanie aplikacji
\end{itemize}

Dzięki zastosowaniu powyzszych technologii system Homelab będzie nowoczesnym, skalowalnym i energooszczędnym rozwiązaniem dla uzytkowników domowych.
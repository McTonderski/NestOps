\chapter{Wprowadzenie}
Współczesna technologia informatyczna umoliwia pasjonatom IT, administratorom systemów oraz programistom budowania i zaradzanie własnymi środowiskami testowymi oraz produkcyjnymi
w domowych warunkach. Koncepcja HomeLab czyli prywatnego środowiska IT, zyskuje na popularności dzięki coraz szerszemu dostępowi do wydajengo sprzętu, technologii wirtualizacji oraz narzędzi do automatyzacji zarządzania infrastrukturą.
Jednak dla wielu uytkowników proces konfiguracji i utrzymania takiego środowiska moze być skomplikowany i czasochłonny.


\section{Cel pracy}
Celem niniejszej pracy magisterskiej jest zaprojektowani i implementacja systemu HomeLab, który uprości proces budowy, konfiguracji oraz zarządzania własną infrastrukturą IT. System ten ma zapewnić uytkownikom intuicyjne narzędza do zarządzania serwereami, maszynami wirtualnymi, kontenerami oraz sieci, a take umoliwić zdalny bezpieczny dostęp do zasobów. Kluczowym załozeniem projektu jest maksymalna automatyzacja procesów, co pozwoli na minimalizację konieczności manualnej konfiguracji i zwiększy wygodę uzytkowania.

\section{Zakres pracy}
W pracy zostaną omówione istotne aspekty techniczne związane z budową homelab, w tym wybór odpowiednich technologii, metod zarządzania infrastrukturą oraz zapewnienia jej bezpieczenstwa. Ponadtoprzedstawiona zsotanie analiza istniejących rozwiązan oraz uzasadnienie wyboru implementowanych funkcjonalnosci. Efektem koncowym pracy będzie gotowy system, który moze zostac wdrozony przez uzytkowników chcących stworzyc wlasne homelab w sposób szybki i efektywny. 

Niniejsza praca stanowi przyczynek do rozwoju narzedzi dedykowanych osobom zainteresowanym budową i zarządzaniem wlasnym srodowiskiem IT, oferując innowacyjne podejście do automatyzacji i ułatwienia dostępu do homelab. W kolejnych rozdziałach zostaną szczegółowo omówione wszystkie kluczowe elementy systemu oraz proces proces jego implementacji.
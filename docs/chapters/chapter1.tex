\chapter{Wprowadzenie}
Wspóczesna technologia informatyczna umoliwia pasjonatom IT, administratorom systemów oraz programistom budowania i zaradzanie własnymi środowiskami testowymi oraz produkcyjnymi
w domowych warunkach. Koncepcja HomeLab czyli prywatnego środowiska IT, zyskuje na popularności dzięki coraz szerszemu dostępowi do wydajengo sprzętu, technologii wirtualizacji oraz narzędzi do automatyzacji zarządzania infrastrukturą.
Jednak dla wielu uytkowników proces konfiguracji i utrzymania takiego środowiska moze być skomplikowany i czasochłonny.


\section{Cel pracy}
Celem niniejszej pracy magisterskiej jest zaprojektowani i implementacja systemu HomeLab, który uprości proces budowy, konfiguracji oraz zarządzania własną infrastrukturą IT. System ten ma zapewnić uytkownikom intuicyjne narzędza do zarządzania serwereami, maszynami wirtualnymi, kontenerami oraz sieci, a take umoliwić zdalny bezpieczny dostęp do zasobów. Kluczowym załozeniem projektu jest maksymalna autoamtyzacja procceesów, co pozwoli na minimalizację konieczności manualnej konfiguracji i zwiększy wygodę uzytkowania.

\section{Zakres pracy}
Opis zakresu badań i problematyki poruszanej w pracy.